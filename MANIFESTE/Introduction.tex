\section{Introduction}

\textbf{LOL C SURTOUT QUE C'EST STYLÉ LE LATEX SUR VSCODE}
\\

\textbf{Description :}
\\
\begin{itemize}
  \item \textbf{Fonction base de données :}
  \begin{itemize}
    \item Un “cahier de bloc” en ligne 
    \item À chaque intervention, l’interne “ajoute une intervention” et remplit plusieurs caractéristiques :
      \begin{itemize}
        \item Laquelle
        \item Conditions (garde ou pas, statut du senior)
        \item Le fait d’avoir fait un geste ou non
        \item Son ressenti et autres données subjectives
        \item Et d’autres données à définir ! 
        \item Un champ libre qui permet d'ajouter un commentaire textuel sur l’intervention et qui pourra être retrouvé facilement dans la fonction visualisation
      \end{itemize}
  \end{itemize}

\vspace{1em}

  \item \textbf{Fonction visualisation :}
  \begin{itemize}
    \item Un onglet permettant de voir des analyses statistiques (script R déjà prêt, à adapter éventuellement en Python)
    \item Un onglet permettant de rechercher des interventions en fonction de critères (ex : toutes les appendicectomies faites en garde avec un senior junior, avec geste, ou avec tel opérateur)
    \item Et donc permettant de retrouver facilement les commentaires textuels associés à ces interventions
  \end{itemize}
\end{itemize}
